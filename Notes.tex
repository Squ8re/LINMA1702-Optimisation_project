\documentclass[french]{article}
\usepackage[T1]{fontenc}
\usepackage[utf8]{inputenc}
\usepackage{lmodern}
\usepackage[a4paper]{geometry}
\usepackage{babel}
\usepackage{enumitem}		% Liste enchainée
\usepackage{amsmath}		% Math
\usepackage{amssymb}		% Math
\begin{document}
\title{Notes et infos}
\author{DAVID Amadéo}
\maketitle
\section{Ce qui coule de source}
\subsection{Hypothèses}
\begin{itemize}
	\item
		L'extraction, la consommation et la circulation de l'eau s'effectue à débit constant.
	\item
		Le débit d'eau extrait aux sources est suffisant pour alimenter toutes les conduites.
\end{itemize}
\subsection{Données}
Le débit d'eau qui s'écoule dans une conduite est donné par l'expression $$\alpha\theta_c\frac{R^2_c\Delta h_c}{L_c}.$$

En plus des données fournies, on aura $L_c$ la longueur de la conduite et $\Theta_c$, un paramètre ajustable entre 0 et 1 représentant une valve.

Données fournies :
\begin{itemize}
	\item 
		Coordonnées en $(x,y,z)$ des points d'approvisionnement et de consommation ($z$ vaut $\Delta h_c$ dans l'équation).
	\item 
		Matrice d'incidence des conduites.
		Il semble que les points intermédiaires aient une somme de 0, les points d'approvisionnement aient une somme de -1 et les points de consommation aient une somme de 1.
	\item
		Rayon des conduites ($R_c$ dans l'équation).
	\item
		Constante de proportionnalité ($\alpha$ dans l'équation).
	\item
		Débits maximaux extractibles aux points d'approvisionnement.
	\item 
		Le coût des débits d'extraction.
	\item 
		Les valeurs minimales (notées $D_{c,min}$) et maximales (notées $D_{c,max}$) du débit en chaque point de consommation.
	\item 
		Le prix facturé aux différents points de consommation.
	\item 
		La matrice d'incidence des points de consommation n'est pas fournie mais peut être calculée facilement et est notée $A_c$.
\end{itemize}

\section{Réponses}
\begin{enumerate}
	\item Analyse d'un réseau existant
	\begin{enumerate}[label=\theenumi.\arabic*]
		\item 
			L'expression $A^T h$ représente les dénivelés entre les conduites.
			L'expression $Af$ représente elle le débit total aux noeuds.
			Notons que $Af$ a comme contrainte d'être plus petit ou égal à $D_{c,max}$ et plus grand ou égal à $D_{c,min}$.
		\item 
			Démonstration :
			
			Un ensemble est défini comme polyèdre si et seulement si c'est un sous ensemble de $R^n$ qui peut être défini comme l'intersection d'un nombre fini de demi-espace de $R^n$. Autrement dit $$P=\left[x \in R^n |Ax\geq b\right].$$
			
			Dans notre cas, l'ensemble $P$ peut être vu comme l'intersection des deux demi-espaces $P_c\geq D_{c,min}$ et $P_c\leq D_{c,max}$.
			Cet ensemble est donc bien un polyèdre.
			
			Vérification d'un profil donné :
			
			Soit un profil $P_c$.
			Pour vérifier qu'il est acceptable, il y a $2n$ contraintes à vérifier avec $n$ le nombre d'éléments du profil.
			Il faut que $P_c[i]\leq D_{c,max}[i]$ et que $P_c[i]\geq D_{c,min}[i]$ avec $i$ allant de $0$ à $n$.
		\item 
			Le problème peut s'écrire tel que
			\begin{align*}
				\max_x P_c^T*P_f - P_a^T * C\\
				P_c\geq D_{c,min}\\
				D_c\leq D_{c,max}\\
				P_A\leq \Delta_{a,max}\\
				\sum P_A - \sum P_C \geq 0.
			\end{align*}
			Avec $P_A$ les débits fournis aux points d'approvisionnement et avec $D_{A,max}$ les débits maximaux extractibles aux points d'approvisionnement.
	\end{enumerate}

	\item Améliorations du réseau
	\begin{enumerate}[label=\theenumi.\arabic*]
		\item TODO
	\end{enumerate}
	\item Conception d'un réseau optimal
	\begin{enumerate}[label=\theenumi.\arabic*]
		\item TODO
	\end{enumerate}
\end{enumerate}
\end{document}
