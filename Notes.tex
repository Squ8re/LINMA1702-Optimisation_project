\documentclass[french]{article}
\usepackage[T1]{fontenc}
\usepackage[utf8]{inputenc}
\usepackage{lmodern}
\usepackage[a4paper]{geometry}
\usepackage{babel}
\begin{document}
\section{Données}
Le débit d'eau qui s'écoule dans une conduite est donné par l'expression $$\alpha\theta_c\frac{R^2_c\Delta h_c}{L_c}$$.
Données fournies :
\begin{itemize}
	\item 
		Coordonnées en $(x,y,z)$ des points d'approvisionnement et de consommation ($z$ vaut $\Delta h_c$ dans l'équation).
	\item 
		Matrice d'incidence des conduites.
		Il semble que les points intermédiaires aient une somme de 0, les points d'approvisionnement aient une somme de -1 et les points de consommation aient une somme de 1.
	\item
		Rayon des conduites ($R_c$ dans l'équation).
	\item
		Constante de proportionnalité ($\alpha$ dans l'équation).
	\item
		Débits maximaux extractibles aux points d'approvisionnement (notons le $D_(c,max)$).
\end{itemize}
\section{Réponses}
\begin{enumerate}
	\item 
		L'expression $A^T h$ représente les dénivelés entre les conduites.
		L'expression $Af$ représente elle le débit total aux noeuds.
		Notons que $Af$ a comme contrainte d'être plus petit ou égal à $D_(c,max)$.
	\item 
		
\end{enumerate}
\end{document}
